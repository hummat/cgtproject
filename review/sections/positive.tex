\documentclass[../review.tex]{subfiles}
\begin{document}
\todo{Name 3 positive points concerning the work, clearly specifying why you think they are well-done or interesting} \textcolor{green}{\Large\checkmark}
\begin{enumerate}
 \item \textbf{Pigou's example:} Introducing a new example from a highly popular source for the same problem encountered in the original work is a good way to revisit the content and to clarify it.
 \item \textbf{Visualiztion:} Using graphs and figures helps a lot when struggling to understand the presented concepts by giving the reader a new perspective from which to approach it.
 \item \textbf{Optimization:} The formulation of a general framework to find and optimal agent-to-path mapping challenges the reader to really interact with the content and thus deepens the understanding.
\end{enumerate}
\end{document}
