\documentclass[../review.tex]{subfiles}
\begin{document}
\todo{Evaluate whether the introduction explains clearly the content of the paper}\\
The abstract should be more elaborated in terms of what the paper is about. Also using the term \textit{agent} both for the entity being routed and a potential supervisory structure (mis)leading the former might be confusing. An interested reader might now, after having read the abstract, have a vague idea of the topics that are about to be discussed. We don't immediately see however why selfishness of agents in a traffic system is a motivation for the study.

The introduction provides a clear overview of topics that will be discussed but contains spelling mistakes, as does the abstract. Since this is your (\todo{direct speech or keep it neutral?}) first impression to the reader, we highly recommend to review the phrasing. Subsequently, the second paragraph states that it is a 'very active topic' but the references that back up this claim come from the same author and date back to 2005. While 2005 is clearly a lot more recent then 1920, the year it is referring to, it would be good to find a more recent paper or remove the 'active topic' claim.

Otherwise the paper is well summarized, touching on all the major parts without getting carried away, so the reader feels confident to dive in. It could be a little clearer though that your work is mainly a reproduction and revisit of the original content rather then something completely novel.
\end{document}
