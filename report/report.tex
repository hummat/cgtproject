\documentclass[letterpaper]{article}
\usepackage{natbib,alifexi}
\usepackage[english]{babel}
\usepackage{blindtext}
\usepackage{xcolor}
\usepackage{microtype}

\newcommand\todo[1]{\textcolor{orange}{Todo: #1}}

\title{Review: Accelerating Multi-agent Reinforcement Learning
with Dynamic Co-learning}
\author{Steve Homer$^1$, Fabian Perez$^1$, Quinten Rosseel$^1$ \and Matthias Humt$^1$ \\
\mbox{}\\
$^1$\{steven.homer, fabian.perez, quinten.rosseel, matthias.humt\}@vub.be}


\begin{document}
\maketitle

\begin{abstract} \label{sec:abstract}
 The article being reviewed introduces a technique to accelerate the learning progress in Multi-Agent Reinforcement Learning (MARL) settings, by dynamically sharing the experience of contextually similar agents with their neighbors.
 \\\todo{Finish abstract}
\end{abstract}

\section{Introduction} \label{sec:introduction}
In the standard Reinforcement Learning (RL) setting, an agent is placed into an unknown environment and provided with a set of actions from which it can choose. By performing an action, the agent can change its state which is then solely defined by the previous state the agent was in before and the chosen action. In certain states, defined by the problem setting, the environment provides feedback to the agent, often called \textit{reward} which can be either positive or negative. From this, the agent begins to approximate the underlying reward function in trying to maximize the expected future reward. Once having started to learn, the agent has to decide whether to exploit current knowledge by choosing the actions it expects to yield the highest reward or to explore new states with potentially higher rewards by trying new state-action combinations. This methodology is typically chosen, if the state-space is large and explicitly defining the correct actions to achieve the goal is either infeasible or even impossible as they are not known. An often cited example in this context is learning to ride a bike \todo{add citation}. The desired outcome is clear, but giving advice how to achieve it in terms of the correct movements to perform is quite difficult.\\
Multi-Agent Reinforcement Learning (MARL) is an extension to the standard RL problem setting, where multiple, often hundreds or thousands of autonomous agents try to pursue their individual goals simultaneously in a common environment. We speak of cooperative MARL, if multiple or all agents pursue the same goal, which means they try to maximize a common reward function. If they succeed, we say that they follow a (near-optimal) joint policy where policy means a mapping from states to actions. To achieve this however has proven to be difficult for large-scale multi-agent systems as it is computationally expensive and requires a large number of update steps until it converges.\\
An important insight on the path to solving this problem is the observation, that an exploitable structure in the problem setting--a distributed load balancing problem, where the agents try to share the work among each other as to minimize processing time--in the form of contextually similar groups of agents emerges during learning.
Real world examples of this effect can be observed in \todo{Add examples}. Agents working on similar tasks under comparable environmental dynamics might benefit from sharing information to accelerate their individual learning progress which is the paradigm proposed in the paper under review. Additionally, strategies to identify promising candidates for information sharing and group formation are proposed, which will be introduced in the next section along with the test setup. In previous work, agents have either been trained individually on their local environment and neighbors, not being able to benefit from the experience of their peers, or as a hive, optimizing a single joint policy, which becomes quickly intractable even for a moderately large number of agents. But identifying contextually similar groups which are likely to greatly benefit from information sharing comes with its own difficulties. What information should be transferred? What does \textit{similar} mean? How well does this approach scale?\\
The authors introduce supervisory agents which are instructed to deal with those questions. A supervisory agent differs from a normal agent only in its ability to receive and act on communications from its subordinates. It forms contextually compatible groups of agents and identifies and shares relevant information between them. We will now take a look at how this is done explicitely.
\\\todo{Add citations}

\section{Methods} \label{sec:methods}

\todo{Explain model as proposed in the paper}
\\\todo{Explain our implementation of the model}

\section{Results}
\todo{Describe the performed simulations and their results}
\\\todo{Include chosen parameter settings}

\section{Discussion}
\todo{Summary and explanation of our work}

\section{Conclusion}
\todo{Add related work and an outlook}
\\\todo{Make sure all questions given below are answered}
\begin{enumerate}
 \item Does the introduction explain clearly the content of the paper
 \item whether there is sufficient background information to understand the relevance of the work
 \item whether the methods are clearly explained (can the results be reproduced?)
 \item whether the results answer the questions asked in the paper.
 \item whether all questions are answered
 \item whether the conclusion is sufficient
 \item and whether the overall style is ok and
 \item whether you believe things are missing in the discussion.
 \item etc.
 \item 3 positive points concerning the work, clearly specifying why you think they are well-
       done or interesting
 \item 3 negative points, which may include missing/unclear explanations or suggestions for
       improvement
 \item at least 3 clear and relevant questions on the content or the methods used which can be asked (next to other questions).
\end{enumerate}

\citep{carroll2005task}, \citep{garant2015accelerating},\\ \citep{ghavamzadeh2006hierarchical}, \citep{gmytrasiewicz2005framework},\\ \citep{guestrin2002multiagent}, \citep{kitano1999robocup},\\ \citep{lazaric2008transfer}, \citep{littman2001value},\\ \citep{nair2005networked}, \citep{nedic2009distributed},\\ \citep{oliehoek2008exploiting}, \citep{price2003accelerating},\\ \citep{renyi1961measures}, \citep{taylor2009transfer},\\ \citep{vickrey2002multi}, \citep{witwicki2010influence},\\ \citep{zhang2010self}, \citep{zhang2013coordinating}

\footnotesize
\bibliographystyle{apalike}
\bibliography{bibliography}

\end{document}
