\documentclass[letterpaper]{article}
\usepackage{natbib,alifexi}
\usepackage[english]{babel}
\usepackage{blindtext}

\title{Review: Accelerating Multi-agent Reinforcement Learning
with Dynamic Co-learning}
\author{Steve Homer$^1$, Fabian Perez$^1$, Quinten Rosseel$^1$ \and Matthias Humt$^1$ \\
\mbox{}\\
$^1$\{steven.homer, fabian.perez, quinten.rosseel, matthias.humt\}@vub.be}


\begin{document}
\maketitle

\begin{abstract}
 The article being reviewed introduces a technique to accelerate the learning progress in Multi-Agent Reinforcement Learning (MARL) settings, by dynamically sharing the experience of contextually similar agents with their neighbors.
\end{abstract}

\section{Introduction}
In the introduction a clear specification of the problem is provided, including related work. Look for additional but relevant references which you can cite in this part.\\[.5cm]

In the standard Reinforcement Learning (RL) setting, an agent is placed into an unknown environment and provided with a set of actions from which it can choose. By performing an action, the agent can change its state which is then solely defined by the previous state the agent was in before and the chosen action. In certain states, defined by the problem setting, the environment provides feedback to the agent, often called \textit{reward} which can be either positive or negative. From this, the agent begins to approximate the underlying reward function in trying to maximize the expected future reward.  Multi-Agent Reinforcement Learning (MARL) is a problem setting, where multiple, often hundreds or thousands of autonomous agents try to pursue their individual goals simultaneously in a common environment. We speak of cooperative MARL, if multiple or all agents pursue the same goal, which is
Agents are characterised by the current state they are in and a set of actions they can perform.

\section{Methods}
In the Methods section you specify how the model is constructed.\\[.5cm]
\blindtext

\section{Results and Discussion}
In the Results section you provide a descriptions of the simulations you have done and their results.  Also include the information concerning the parameter settings.\\
In the Discussion section you provide a summary and an explanation of your work.\\[.5cm]
\blindtext

\section{Conclusion}
Related work. All articles cited in the original paper.\\[.5cm]
\citep{carroll2005task}, \citep{garant2015accelerating},\\ \citep{ghavamzadeh2006hierarchical}, \citep{gmytrasiewicz2005framework},\\ \citep{guestrin2002multiagent}, \citep{kitano1999robocup},\\ \citep{lazaric2008transfer}, \citep{littman2001value},\\ \citep{nair2005networked}, \citep{nedic2009distributed},\\ \citep{oliehoek2008exploiting}, \citep{price2003accelerating},\\ \citep{renyi1961measures}, \citep{taylor2009transfer},\\ \citep{vickrey2002multi}, \citep{witwicki2010influence},\\ \citep{zhang2010self}, \citep{zhang2013coordinating}

Questions to be answered by reviewing party:
\begin{enumerate}
 \item Does the introduction explain clearly the content of the paper
 \item whether there is sufficient background information to understand the relevance of the work
 \item whether the methods are clearly explained (can the results be reproduced?)
 \item whether the results answer the questions asked in the paper.
 \item whether all questions are answered
 \item whether the conclusion is sufficient
 \item and whether the overall style is ok and
 \item whether you believe things are missing in the discussion.
 \item etc.
 \item 3 positive points concerning the work, clearly specifying why you think they are well-
       done or interesting
 \item 3 negative points, which may include missing/unclear explanations or suggestions for
       improvement
 \item at least 3 clear and relevant questions on the content or the methods used which can be asked (next to other questions).
\end{enumerate}

\footnotesize
\bibliographystyle{apalike}
\bibliography{bibliography}

\end{document}
